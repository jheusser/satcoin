\documentclass[conference]{IEEEtran}
% packages
\usepackage{xspace}
\usepackage{hyperref}
\usepackage{todonotes}
\usepackage{tikz}
\usepackage[utf8]{inputenc}

%  commands
% \newcommand{\todo}[1]{$\langle\!\langle$\marginpar[\raggedleft$\triangleright\triangleright\triangleright$]{$\triangleleft\triangleleft\triangleleft$}\textsf{#1}$\rangle\!\rangle$}
\def\CC{{C\nolinebreak[4]\hspace{-.05em}\raisebox{.4ex}{\tiny\bf ++}}}
\def\ea{\,et\,al.\ }

\begin{document}
	
% paper title
\title{SATcoin -- Bitcoin mining via SAT}

% author names and affiliations
% use a multiple column layout for up to three different
% affiliations
\author{
\IEEEauthorblockN{Norbert Manthey}
\IEEEauthorblockA{nmanthey@conp-solutions.com\\Dresden, Germany}
\and
\IEEEauthorblockN{Jonathan Heusser}
\IEEEauthorblockA{jonathan.heusser@gmail.com\\London, England}
}

\maketitle

\def\minisat{\textsc{Minisat~2.2}\xspace}
\def\cbmc{\textsc{CBMC}\xspace}

% the abstract is optional
\begin{abstract}
The main security properties of Bitcoin, censorship resistance and an immutable historical ledger of transactions is enforced by its mining algorithm.
By design this "proof of work" requires a large amount of computational power to verify transactions.
While current mining algorithms are based on brute force, this document briefly describes how the same problem can be solved via SAT.
Most details of this document are taken from~\cite{satcoin-online-article}.
\end{abstract}

\section{Bitcoin Mining}

A Bitcoin mining program essentially performs the following (in pseudo-code):

\begin{verbatim}
nonce = MIN
while(nonce < MAX):
  if sha(sha(block+nonce)) < target:
     return nonce
  nonce += 1
\end{verbatim}

The task is to find a nonce which, as part of the bitcoin block header, hashes below a certain value.

This is a brute force approach to something like a preimage attack on SHA-256.
The process of mining consists of finding an input to a cryptographic hash function which hashes below or equal to a fixed target value.
At every iteration the content to be hashed is slightly changed to find a valid hash; there's no smart choice in the nonce.
The choice is essentially random as this is the best one can do on such hash functions.

In~\cite{satcoin-online-article}, Heusser proposed an alternative mining algorithm which does not perform a brute force search.
Instead, we utilize tools from the program verification domain to find bugs or prove properties of programs, see as example~\cite{Heusser:2010:QIL:1920261.1920300}.
To find the correct nonce or prove the absence of a valid nonce, a model checker backed by a SAT solver is used on a C implementation of the hashing.
In contrast to brute force, which actually executes and computes many hashes, the new approach is symbolically executing the hash function with
added constraints that are inherent in the bitcoin mining process.
The submitted benchmark is based on CNFs created by CBMC.
    
\section{Bitcoin mining using SAT Solving and Model Checking}

We take an existing C implementation of SHA-256 from a mining program and strip away everything but the actual hash function and the basic mining procedure of sha(sha(block)).
This C file is the input to CBMC~\cite{DBLP:conf/tacas/KroeningT14}.

By adding the right assumptions and assertions to the implementation, i.e. MIN and MAX in the pseudo code above, we direct the SAT solver to find a nonce.
Instead of a loop which executes the hash many times and a procedure which checks if we computed a correct hash,
we add constraints that when satisfied implicitly have the correct nonce in its solution.

The assumptions and assertions can be broken down to the following ideas:
% 
\begin{itemize}
 \item The nonce is modelled as a non-deterministic value
 \item The known structure of a valid hash, i.e. leading zeros, is encoded as assumptions in the model checker
 \item An assertion is added stating that a valid nonce does not exist 
\end{itemize}
% 
More details about the translation, and a basic solver preformance comparison can be found in the full article~\cite{satcoin-online-article}.

\section{SAT Competition Specifics}

The used input file for CBMC implements the described analysis, and uses the so called \emph{genesis block} of the bitcoin blockchain, the very first block of the chain.
We provide a script that allows to produce a CNF based on the fact whether the formula should be satisfiable, or unsatisfiable.
The formula is satisfiable, if the valid nonce is part of the given range.
Hence, the range is constructed such that the valid nonce is right in the middle of the range to be analyzed.
For unsatisfiable formulas, the range starts right after the valid nonce.

The code, as well as all scripts to create a benchmark are available at \url{https://github.com/jheusser/satcoin}.

% \section{Availability}
% 
% The source of \cbmc can be found at \url{https://github.com/diffblue/cbmc}.
% For more details, please have a look at \url{http://www.cprover.org/cbmc/}.

\bibliographystyle{IEEEtran}
\bibliography{local}

\end{document}